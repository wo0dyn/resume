
%
% Curriculum Vitæ/Résumé
% Nicolas Dubois <nicolas.c.dubois@gmail.com>
% http://nicolasdubois.com • @wo0dyn
%

\documentclass[a4paper, 11pt]{scrartcl}

% --------------------------------- Packages
\usepackage[american, francais]{babel}
\usepackage[utf8]{inputenc}
\usepackage[pdftex]{xcolor}

\usepackage{fancyhdr}
\usepackage{graphicx}
\usepackage{url}
\usepackage{xspace}

\usepackage{packages/vendor/aeguill}
\usepackage{packages/vendor/vmargin}

\urlstyle{sf}

\newcommand\email{\begingroup\Url}

\usepackage[pdftex, colorlinks,%
	pdfpagelayout=SinglePage,%
	pdftitle={Curriculum Vitæ — Résumé},%
	pdfauthor={Nicolas Dubois, @wo0dyn},%
	pagebackref=true,%
	pdfnewwindow=true,%
	bookmarksnumbered=true,%
	pdfstartview={FitH},
	urlcolor=black,
	linkcolor=black,
	bookmarksopen]%
{hyperref}

\pagestyle{empty}

\setmarginsrb
        {10mm} % est la marge gauche
        {10mm} % est la marge en haut
        {10mm} % est la marge droite
        {10mm} % est la marge en bas
        {00mm} % fixe la hauteur de l'entête
        {00mm} % fixe la distance entre l'entête et le texte
        {00mm} % fixe la hauteur du pied de page
        {00mm} % fixe la distance entre le texte et le pied de page
% note : utilisation du package vmargin.

% --------------------------- Redéfinition des sections :
\let\mySection\section
\renewcommand{\section}[1]{
\addcontentsline{toc}{section}{#1}%
\mySection*{#1 \hrulefill}%
}

% --------------------------- Redéfinition des subsections :
\let\mySubsection\subsection
\renewcommand{\subsection}[1]{%
\addcontentsline{toc}{subsection}{#1}%
\mySubsection*{\hspace{5mm}$\blacktriangleright$ #1}%
}

% --------------------------- Redéfinition des items :
\let\myItem\item
\renewcommand{\item}[1]{%
%\myItem \textbf{\textsf{#1}}
\myItem {\titlefont #1}%
}

% --------------------------- Nouvelles commandes :
% Citer un site web :
\newcommand{\website}[1]{\ \\\hspace{35mm}{\large \Mundus} \email{#1}}
% Citer un boulot :
\newcommand{\workItem}[6]{%
\item{#1}, #2 (#3), #4, #5%
\begin{quote}%
\hspace{-7mm}%
#6
\end{quote}%
}

% Mise en page de l'état civil :
\newlength\width%
\newcommand{\etatCivil}[1] {%
\width=\textwidth\advance\width by -2cm%
\advance\width by -.5cm%
\begin{minipage}[c]{32mm}
\ifpdf\includegraphics[width=30mm]{images/wo0dyn-circle.jpg}%
\else \includegraphics[width=30mm]{img/Nicolas-Dubois.eps}\fi%
\end{minipage}\hfill%
\begin{minipage}[c]{\width}#1\end{minipage}%
}


\begin{document}

\sf

\begin{center}
	{\titlefont\Huge Curriculum Vitæ $\cdot$ Nicolas Dubois}\\
	{\titlefont\hspace{2cm}36 ans, développeur d'applications web, \email{me@nicolasdubois.com}}
\end{center}

\section{État civil}
\hspace{10mm}
\etatCivil{

\begin{tabular}{p{48mm}@{ }p{80mm}}
    {\bfseries Situation de famille}\dotfill&Divorcé, 2 enfants\\
    {\bfseries Nationalité}\dotfill&Française\\
    {\bfseries Date de naissance}\dotfill&15 juin 1979 à Nancy (54)\\
    {\bfseries Page personnelle}\dotfill&\email{http://nicolasdubois.com}\\
    {\bfseries Moyens de locomotion}\dotfill&Permis B, véhicule\\
    {\bfseries Centres d'intérêts}\dotfill&Musique (guitare), sports (basket-ball, golf, ski)\\
\end{tabular}
}

\section{Expérience professionnelle}

\subsection{Emplois en informatique}

\begin{description}

    \workItem{AMG Développement | Groupe GPdis}{Eurocentre}{31}{CDI}{depuis Octobre 2009}{
        Je suis actuellement développeur d'applications web chez AMG Développement, pôle
        e-commerce de la filiale du groupe GPdis.
        J'interviens sur les applications internes (Django, symfony, ZF) et également sur
        la plateforme Magento du groupe (Discounteo, Villatech, Pulsat).
        \website{http://www.villatech.fr}
    }

    \workItem{Ekinos | Groupe Eklas}{Labège}{31}{CDI}{Avril 2009--Octobre 2009}{
        J'ai été développeur/concepteur web et chef de projet junior pour la société Ekinos,
        une agence web spécialisée dans le développement de solutions e-commerce avec la
        plateforme Magento.
        \website{http://www.labsoft.fr}
    }

    \workItem{WaterProof}{Labastide Saint-Pierre}{82}{CDI}{Novembre 2007--Avril 2009}{
        J'ai été développeur/concepteur web pour la société WaterProof, qui développe et commercialise
        l'IDE PHPEdit, où j'ai participé au développement d'applications web à forte valeur ajoutée pour
        des grands comptes (Spot Image, Dalta, ITM) de la région toulousaine avec le framework symfony\footnote{\sffamily J'ai
        suivi la formation symfony dispensée par SensioLabs, créateur du framework.}.
        \website{http://www.waterproof.fr}
    }

    \workItem{MaisMoinsCher.com}{Gaillac}{81}{CDI}{Novembre 2005--Octobre 2007}{
        J'ai été développeur web principal pour le e-commerçant MaisMoinsCher.com
        où j'ai développé :\\
            \hspace*{1cm}- une application intranet pour la gestion du catalogue de la boutique ;\\
            \hspace*{1cm}- plusieurs modules dans la partie backoffice dont notamment un gestionnaire de stock ;\\
            \hspace*{1cm}- un module de paiement Sofinco pour la partie frontoffice ;\\
            \hspace*{1cm}- un pricebot (outil de veille concurrentielle) pour la gestion des fournisseurs.
        \website{http://www.maismoinscher.com}
    }

    \workItem{Les cahiers de la cypriolette}{Commecy}{55}{CDD}{Août 2005}{
        Développement de scripts pour interfacer une base de données MySQL avec une application Flash.
        \website{http://www.cypriolette.com}
    }

    \workItem{Laboratoire Leibniz--IMAG}{Grenoble}{38}{Stage Master de Recherche}{Février--Juillet 2005}{
        Développement et intégration d'un module émotionnel sur une plateforme d'agent conversationnel de recommandation.
        Modélisation en UML et implémentation en langage Java.%\\
        %$\bullet$ Directrice : Sylvie Pesty (PR - Université Pierre Mendès-France)
        \website{http://magma.imag.fr}
    }

    \workItem{Université du Luxembourg}{Walferdange}{Luxembourg}{Stage non obligatoire}{Juillet 2004}{
        Réalisation de formats d'items dynamiques déployés par la plateforme de testing assisté
        par ordinateur au sein de l'équipe de recherche EMACS%\footnote{\sf EMACS: Educational
        %Measurement and Applied Cognitive Science}.%\\
        %$\bullet$ Encadrant : Romain Martin (PR - Université du Luxembourg)
        \website{http://www.tao.lu}
    }

        % \workItem{Valoria}{Vannes}{56}{Stage de Maîtrise}{Avril--Mai 2004}{%
        % Évaluation quantitative par implémentation de fichiers logs dans l'application Sibylle, à l'aide du
        % langage XML ; puis développement d'un utilitaire Java pour analyser les fichiers logs.%\\
        % %$\bullet$ Encadrants : Franck Poirier (PR - UBS) et Igor Schadle (ATER - UBS)\\
        % %$\bullet$ Tuteur universitaire : Anna Maria Berardi (MCF - Metz)
        % \website{http://www-valoria.univ-ubs.fr}
        % }

        %\workItem{LORIA}{Vand{\oe}uvre-lès-Nancy}{54}{Vacation CNRS}{Décembre 2003}
        %{Réalisation de documentations, en anglais, relatives à l'API FeatureStructure développée au LORIA.}


       % \workItem{Juillet-Août 2003}{Stage non obligatoire}{LORIA\footnote{\sf LORIA : laboratoire Lorrain de Recherche en Informatique et ses Applications.} : Vand{\oe}uvre - 54}
        %{Implémentation d'un analyseur syntaxique de structures de traits en langage XML conforme à la
        %DTD~ISO~TC~37/SC4. À l'issu de ce stage, mon mémoire a servi de document pour la délégation française
        %de proposition de norme pour la représentation de structure de traits. Création de pages Web pour la
        %distribution de l'API FS.\\
        %$\bullet$ Tuteur : Azim Roussanaly (MCF - Nancy 2)
        %\website{http://pauillac.inria.fr/atoll/RNIL/FSR.html}
        %\website{http://www.loria.fr/~azim/FS}
        %}

        %\workItem{Juillet-Août 2002}{Stage non obligatoire}{LORIA\footnotemark[3] : Vand{\oe}uvre - 54}
        %{Extension du formalisme structures de traits aux disjonctions sur valeurs dans l'API FS développée au LORIA
        %et réécriture d'un analyseur syntaxique à l'aide de l'outil JavaCC.\\
        %$\bullet$ Tuteur : Azim Roussanaly (MCF - Nancy 2)
        %\website{http://led.loria.fr/en_outils.php}
        %}
\end{description}

\subsection{Enseignement}
\begin{description}
        \workItem{Vidéoscop de Nancy}{Nancy}{54}{Vacation}{Mars--Juin 2003}{
            Médiatisation du cours d'Interface Homme-Machine de Kamel Smaïli (PR - Nancy 2) pour le projet e-miage.
            Réalisation de supports multimédia et d'expériences à l'aide des outils Flash et PHP.
            \website{http://www.e-miage.org}
        }

        \workItem{Université Nancy 2}{Nancy}{54}{Tuteur}{Février--Juin 2002}{
            Tuteur en informatique à la faculté d'AES -- Initiation aux outils de bureautique Word et Excel.
            \website{http://www.univ-nancy2.fr/VIDEOSCOP}
        }
\end{description}

\section{Formation}

\subsection{Diplômes}

\begin{description}
	\item{2004--2005 : Master de Recherche \og{}Information, Cognition et Apprentissages\fg{}}, spécialité Sciences Cognitives
	obtenu à l'Institut National Polytechnique de Grenoble, mention AB
        \begin{quote}
                \textbf{Mémoire} : \emph{modélisation informatique de la prise de décision à base d'émotions.}
        \end{quote}
        \item{2003-2004 : Maîtrise de Sciences Cognitives} (spécialité Informatique) obtenue à l'Université Nancy 2, mention AB
        \begin{quote}
                \textbf{Travail d'Études et de Recherche} :
                \emph{état de l'art des systèmes de communication assistée par ordinateur.}%, point de vue ergonomie cognitive et ingénierie des langues}
        \end{quote}
        \item{2002-2003 : Licence de Sciences Cognitives} obtenue à l'Université Nancy 2
        \item{2000-2001 : DEUG MIAGE} obtenu à l'IUP MIAGE\footnote{\sffamily IUP MIAGE : Institut Universitaire Professionnalisé en Méthodes Informatiques Appliquées à la Gestion des entreprises.} de l'Université Nancy 2
        \item{1998-1999 : Baccalauréat Scientifique} (spécialité Physique-Chimie) obtenu au Lycée Henri Poincaré de Nancy
\end{description}

\section{Compétences}

\subsection{Compétences informatiques}

\begin{description}
        \item{Environnements : }\textbf{GNU/Linux} (Ubuntu, Debian), Unix (Solaris, SunOS), MacOSX ;
        \item{Langages : }\textbf{Python}, Unix Shells (bash/zsh), Java, {\LaTeX}, XML, XSL ;
        \item{Développement Web : }\textbf{Python}, \textbf{HTML} 5, (S)\textbf{CSS}, JavaScript, PHP ;
        \item{Bases de Données : }\textbf{PostgreSQL}, MySQL, SQLite ;
	    \item{Applications/Frameworks : }\textbf{Django}, flask, symfony (1.x), Magento.
\end{description}

\subsection{Compétences linguistiques}

\textbf{Français : }langue maternelle \textopenbullet{} \textbf{Anglais : }lu, écrit (technique) \textopenbullet{} \textbf{Espagnol : }notions.

\end{document}
